\begin{abstract}
与三维模型形变有关的相关技术一直许多研究关注的问题,
在动画制作和游戏开发等领域也有着广泛的应用。
而随着近年来消费级深度相机的的普及,
使得基于深度相机研究与应用越来越多。

本文提出了一种基于RGBD视频的形变子空间技术。
本文以RGBD相机采集的RGBD视频为输入,
实现了从三维模型重建、物体位姿估计、形变捕捉、形变子空间构建的流程。
基于形变子空间的模型修改方法能够保证修改的模型符合物体本身的形变特征。
本文的工作可主要分为四步:
首先,本文会用深度相机扫描物体,
捕捉物体表面的形状信息,
并据此重建三维模型。
然后本文会以RGBD图像为输入,
根据操作者的手部信息分割出属于深度图像中属于物体的部分,
并据此估计出物体的初始位姿。
这一步可看做形变捕捉的预处理。
然后本文会以深度视频为输入,
用基于双四元数的形变图描述模型的形变,
通过优化形变图参数捕捉物体的形变,
并挑选出有代表性的形变状态作为形变关键帧。
最后本文会从形变关键帧中提取出基于形变梯度的特征向量,
以这些特征向量作为基向量张成非线性的特征空间。
特征空间中的特征向量对应的形变模型构成了模型的形变子空间。
此外本文还给出了基于形变子空间的模型形状修改方法,
保证修改的模型尽可能落在形变子空间内。

由于本文的三维模型和形变关键帧都是借助RGBD相机捕捉物体的表面信息获得,
并不是通过手工的编辑获得,
所以能保证静态的三维模型和形变关键帧都是符合物体本身的特点的。
所以由此构建出的形变子空间是较为合理的。

\keywords{RGBD相机 ,双四元数,形变捕捉,形变子空间}
\end{abstract}
