
\section{本文工作}
本文借助了深度相机,以深度视频作为输入,
完成了一个从三维模型重建、模型形变捕捉到模型形变子空间构建的流程。
深度视频由按照时间顺序排列的深度图像构成。
与普通的图像不同,深度图像将场景中采样点距图像采集器的距离,即深度,作为其像素值,
它能够直观的描述场景的几何信息。
基于深度图像中记录的几何信息,本文实现了三维模型的重建以及模型形变的捕捉,
并利用捕捉到的形变信息,构建了模型的合理的形变子空间,
使得模型在被修改的过程中发生的形变始终落在这个子空间内。

