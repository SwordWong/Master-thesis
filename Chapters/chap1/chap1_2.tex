
\section{本文工作}
本文借助了深度相机,以深度视频作为输入,
完成了一个从三维模型重建、模型形变捕捉到模型形变子空间构建的流程。
深度视频由按照时间顺序排列的深度图像构成。
与普通的图像不同,深度图像将场景中采样点距图像采集器的距离,即深度,作为其像素值,
它能够直观的描述场景的几何信息。
基于深度图像中记录的几何信息,本文实现了物体的三角网格模型的重建以及模型形变的捕捉,
并基于捕捉到的形变信息,构建了模型的合理的形变子空间,
使得模型在被修改的过程中发生的形变始终落在这个子空间内。

本文利用深度相机扫描物体,基于扫描过程中采集的深度信息重建三维模型。
深度图像记录了物体表面上的采样点与相机的相对位置。在扫描的过程中,本文会跟踪相机的运动轨迹,
得到每一帧的相机位置,从而计算出物体在全局坐标系中的位置。
在重建的步骤中,本文用体素描述三维模型,并根据每一帧深度图像更新体素的值。
体素模型可通过Matching Cubes\cite{lorensen1987marching}算法转换为三角网格模型。

在获得了物体的静态三维模型后,本文需要用户摆弄物体使之发生形变,
并利用深度相机捕捉物体的形变,得到模型的形变关键帧。
在形变捕捉阶段,本文以静态三维模型与记录了物体形变过程的深度视频作为算法的输入,
得到一组形变后的模型作为该模型的形变关键帧。
本文用Deformation Graph\cite{sumner2007embedded}描述模型的形变。
对于每一帧,本文会优化Deformation Graph的参数,令形变后的模型和深度图像尽可能吻合,
从而得到发生形变后的模型。
本文会从每一帧的形变模型中挑选其中具有代表性的模型作为模型的形变关键帧。

然后,本文会从形变关键帧中提取出特征向量,构建模型的形变子空间。
从形变关键帧中提取的的特征向量描述了形变模型相对于静态模型的相对形变。
当静态模型已知时,特征向量和形变模型可以互相转换。
从形变关键帧中提取的特征向量称为关键向量,由特征向量张成的空间称为特征空间,即模型的形变子空间。
用户可以通过修改少数控制点修改模型,修改后的模型所对应的特征向量必定落在模型的形变子空间内。
由于形变关键帧是通过捕捉现实中物体的形变得到,所以所有的形变关键帧都是合理的,
由关键向量张成的形变子空间也是较为合理的。

