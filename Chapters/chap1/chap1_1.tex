
\section{课题背景}
三维模型的获取和编辑方面的研究有着悠久的历史,在工业界也有着十分广泛的应用。
随着近年来游戏、动画产业的高速发展,工业界对于三维模型的需求量也日益增加。
因此,如何高效的获取三维模型,如何有效的编辑三维模型的姿态逐渐成为学术界和工业界共同关注的问题。

三维模型的构建过程通常称为三维建模。
对于现实世界中已经存在的物体,业内通常通过三维扫描的方式建模。
搭载深度摄像头的三维扫描设备不仅能获取视野中的颜色信息,还能通过三维测距技术获取深度信息。
传统的三维扫描设备价格昂贵、操作复杂,通常在较为专业的领域使用。
近年来,消费级深度相机越来越普及,Microsoft、Google、Apple等科技巨头也纷纷推出了搭载了深度相机的产品。
这使得通过三维扫描获得三维模型越来越容易,基于深度的信息的相机的应用场景也越来越多。

在游戏、动画等领域,仅仅用一个静态的三维模型描述一个物体是远远不够的,
它们需要以许多不同的姿态、形状出现。
显然,通过修改原有模型生产新的三维模型可以大幅降低建模成本。
使三维模型改变姿态、形状,需要修改其三维信息。
以多边形网格模型为例,需要修改顶点的三维坐标。
直接修改网格模型的每个顶点的位置显然不现实,
现有的模型修改工具通常通过修改少数控制点使模型在符合一定约束的情况下发生形变,从而得到新的模型。
在修改模型的过程中,可能会使模型产生不符合物体特性的形变。
如何方便的修改模型,如何保证模型在修改的过程中发生的形变尽量合理一直是业内关注的问题。
之前出现过很多数据驱动的形状修改方法,
如动画和游戏领域常用的Blend Shape和PCA技术。
这些方法都有各自的局限。
这些方法用已知的形状创造出新的形状。
但在传统的数据驱动的形状修改方法中的形变样例通常也是通过手工编辑得到的,
同样存在形变结果不符合模型形变特征的可能。

除了模型形状编辑技术,
学术界还有另一种改变模型形状的思路,
就是形变捕捉技术。
随着深度相机的普及,
近年来有很多借助深度信息捕捉物体形变的相关研究。
通过形变捕捉得到的模型的形状是和深度相机捕捉的信息相符合的,
所以由该技术获得的形变模型都是符合物体本身的形变特性的。

本文结合了三维重建、形变捕捉和数据驱动的形状修改技术,
提出了一种基于RGBD视频的形变子空间构建技术,
能够让模型发生符合物体形变特性的形变。