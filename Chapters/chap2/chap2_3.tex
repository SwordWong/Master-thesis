\section{三维模型形变}
三维模型的形变一直是图形学届所关注的研究课题。
形状建模的早期工作主要聚焦在模型的空间形变,
Barr的工作\cite{barr1984global}就提供了全局的空间重映射。
FFD(Free-form deformation,自由形变)\cite{sederberg1986free}用三维晶格将空间形变参数化,
提供了一个高效的对复杂形状施加粗略形变的方法。
但要得到高质量的形变效果,需要更多细节的控制点\cite{coquillart1990extended},
也会引入更多的手工操作。
常见的网格模型修改工具允许用户移动少量顶点让模型发生形变,同时要保留模型的细节和平滑。
细分和多分辨率技术将网格细节编码为顶点的在拓扑上的
偏移\cite{zorin1997interactive}\cite{kobbelt2000multiresolution}
或者几何上更为简单的基网格\cite{kobbelt1998interactive},
从而在不同的尺度下达到细节保留的目的。

另一些修改网格的模型的方法则采用了拉普拉斯坐标或金字塔坐标等本征表达。
在这些方法中,每个顶点的位置是由它和相邻节点的关系编码的,
所以局部的改动可以通过本征表达在网格重建的过程中传播的周围的节点。
Yu和Zhou的工作\cite{yu2004mesh}求解了离散在整个网格上的泊松方程。
Sumner提出了形变梯度(deformation-gradient)\cite{sumner2004deformation}来描述形变。
形变梯度描述了三角形网格模型中每一个三角面片相对于参考模型(reference mesh)中对应面片的仿射变换。
Sumner后来提出的Mesh-Based IK\cite{sumner2005mesh}也采用了这种描述形变的方式构建形变子空间,
本文构建子空间采用的也是这一方法。
