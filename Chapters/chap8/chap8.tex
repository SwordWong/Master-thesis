\chapter{总结与展望}
模型形状的修改在工业界有着广泛的应用。
本文提出了一个以RGBD视频为输入,从模型重建、形变捕捉到形变子空间构建的算法,
并给出了基于子空间的模型形状修改方法。
由于本文的模型通过扫描重建,
用于构建子空间的形变关键帧由形变捕捉得到,
形变关键帧一定是现实中物体会发生的,
所以由此构建的形变子空间是较为符合物体本身形变的特性的。

本文的算法也有着一些限制和可以进一步改进的方案。
首先,由于本文在形变捕捉阶段使用的形变描述方法会根据相邻节点约束传递形变,
本文的方法适合能捕捉平滑的柔性形变。
对于一些不连续的突变的形变,就不能很好的处理,如铰接件、抽屉等。
对于这种情况,不能用通用的柔性形变模型描述,
可以将模型分割为多个发生刚体变换的部分再各自跟踪各自发生的变换。

此外,本文的形变捕捉流程不能捕捉细微的形变。
这是因为本文形变捕捉方法中,
顶点的变换由形变图节点的形变插值而成,
又有相邻节点间的约束,
所以这种形变的描述方式描述不了局部的细小形变,
如物体表面的褶皱。
此外,本文使用的Kinect一代深度相机采集的深度数据精度较低且有较大的噪声,
无法准确捕捉物体表面的细小形变。
如需捕捉这些细小的形变,
可以使用精度更高的深度相机采集物体表面的深度信息。
此外可以用基于形变图的形变捕捉方法捕捉方法物体的大致形变姿态,
在用直接优化局部顶点偏移的方法获取局部的细小形变。

最后,本文在形变捕捉阶段用于形变图描述形变,
但在形变子空间构建时用每个面片的形变梯度描述模型的形变,
两个阶段描述形变的精度与形式不匹配。
针对这个问题有两个角度的改进策略。
一是在形变捕捉阶段,可以用更底层的方式,如顶点和面片的信息,描述形变。
二是直接根据形变图提取形变特征向量,构架基于形变图的形变子空间。

在未来的工作中,我们希望能够进一步拓展算法,
使之可以应用于更广泛的使用场景,
并用有高质量的结果。
